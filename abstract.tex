\newpage
\begin{center}
\thispagestyle{empty}
\vspace*{2\baselineskip}
\LARGE{\textbf{ABSTRACT}}\\[0.5cm]
\end{center}
\thispagestyle{empty}
\doublespacing
\begin{normalsize}
The core objective of the project is to implement a smart traffic management system, that reduces the traffic congestion in large cities using IOT systems. The system measures the congestion of a given lane in real time and adjust accordingly the time period for which the green signal is to be turned on for that lane.

The traffic congestion of each lane in a signal is monitored continuously, using Ultrasonic sensors. This data is processed using RPi, using which the ON time of the green signal of a particular lane is estimated, by an internal algorithm.

The system also consists of a manual override, which can be used by the traffic controller, to override the automatic smart traffic system, in case of emergencies like - to allow ambulances, fire brigades, etc. through cloud-based application. It can also be toggled into NORMAL mode that operates similar to the conventional legacy systems.

This IOT based smart traffic management system overcomes the flaws in the existing traffic management system, by means of providing an adaptive control of the traffic, with respect to the congestion at real time.
\\[1cm]
\end{normalsize}