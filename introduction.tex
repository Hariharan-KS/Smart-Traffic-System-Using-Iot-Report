\chapter{Introduction}

\section{Overview}
Smart traffic management system is an automated way of controlling and co ordinating the traffic in the city, in accordance with the real time traffic analytics, thus reducing the traffic congestion. In large Metropolitian cities, traffic congestion is a major problem. The congestion is caused due to number of factors like, increase in the number of vehicles that exceeds the capacity of the road, inefficient traffic management system, etc.

The current traffic management system is manual/preset and has a number of limitations, such as,
\begin{itemize}
\item Inability of the system to tackle the ever-increasing number of vehicles at a rapid rate.
\item Inefficient setup to co-ordinate vehicles like ambulances with minimum delay, through traffic, in case of emergencies.
\item Delay in the clearance of traffic due to rigid setup of the legacy systems, etc.
\end{itemize}

This IOT based smart traffic management system overcomes the flaws in the existing traffic management system, by means of providing an adaptive control of the traffic, with respect to the congestion at real time.

\pagebreak

\section{Problem Statement}
In the current traffic systems, the time for which the vehicles are allowed to pass through in a lane in the signal is fixed. This method is inefficient to handle the ever-growing traffic. Also, the traffic congestion of the cities is highly variable and unpredictable, which proves it difficult to preset these fixed timers of the traffic signals. Thus, it cannot handle the congestion of vehicles in the city, in an effective way. It can lead to waste of time when there is very low congestion and might build up the traffic during peak times.

Dipak K Dash \cite{trafficcongestion} has, it is estimated that, in our country, due to congestion, there is an annual loss of around \rupee60,000 crores which includes fuel wastage. Due to the traffic congestion, the average fuel mileage of a vehicle is only about 3.9kmpl and it also leads to wastage of time.

The proposed model is flexible and adaptive, which can handle these dynamic traffic trends. The model measures the current traffic of any given lane in real time and depending upon this value, it adjusts or adapts the time for which the green signal is to be turned ON/OFF, thereby ensuring smooth flow of vehicles, reducing congestion.

\section{Project Objective}
The core objective of the project is to implement a smart traffic management system, that reduces the traffic in large cities using IOT systems. The system measures the congestion of a given lane in real time and adjust accordingly the time period for which the green signal is to be turned on for that lane.

It also includes features like, prioritizing the movement of emergency vehicles like, ambulances, fire brigades, preventing the violation of traffic rules, etc. through various automation systems.

\pagebreak

\section{Literature Survery}
A continuous increase in the number of vehicles being used by the masses, put forth an issue of increased traffic and congestion, which is to be addressed. Daily commute of people in and around the city, has been a problem due to inefficient management of the traffic in the city.

According to this article \cite{trafficcongestion}, it is estimated that, in our country, due to congestion, there is a annual loss of around Rs.60,000 crores which includes fuel wastage. Due to the traffic congestion, the average fuel mileage of a vehicle is only about 3.9kmpl and it also leads to wastage of time.

Here, Hussain T.M, et.all \cite{infrared}, describes an experimental infrared optical system that was designed to detect and monitor vehicular road traffic. The system used was developed and 1st tested in the laboratory using infrared laser sources and detectors in conjunction with computerized signal processing and correlation techniques. Preliminary road tests confirmed the system's capability to detect, monitor, and count the passage of vehicular road traffic.

In this paper \cite{multiloop}, S.S.M Ali presents an inductive loop vehicle detection system suitable for heterogeneous and less-lane disciplined traffic. A multiple loop system that is suitable for sensing vehicles in a heterogeneous and less-lane disciplined condition. Each loop has a unique resonance frequency, when a vehicle goes over a loop, the corresponding inductance and resonance frequency will change. This shift in frequency can be used to detect vehicles.

Xianbin Cao, et.all \cite{lowaltitude}, discusses about visual surveillance from low-altitude airborne platforms, that plays a key role in urban traffic surveillance. Moving vehicle detection and motion analysis are very important for such a system. This paper has two major contributions: First, to speed up feature extraction and to retain additional global features for higher classification accuracy. Second, to efficiently correlate vehicles across different frames for vehicle motion trajectories computation, thus achieving better performance with higher detection rate, lower false positive rate, and faster detection speed.

Based on the references from the above papers, we have come up with a more, simple model to control traffic and congestion effectively and co-ordinate emergency vehicles route as mentioned in the objective of the project.
 % adds the Literature Survey page

\section{Organization of Report}
This report is organized as follows,
\begin{itemize}
\item Chapter 1: Introduction - briefing about the overview, literature survey, problem formulation and methodology of the project.
\item Chapter 2: System Design – Explains the overall system architecture and the different components used in the system.
\item Chapter 3: Working principle and Procedure - It explains about the working principle of the project and the procedure to setup and run the model
\item Chapter 4: Result and Output – It consists of the summary of the results/output observed accompanied with pictures and associated discussions.
\item Chapter 5: Conclusion and Future Scopes – Finally, it is concluded with the project’s pros and cons, with the future possibilities in which the model can be improved.
\end{itemize}
