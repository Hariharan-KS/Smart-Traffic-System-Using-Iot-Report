\chapter{Conclusion and Future Works}

\section{Conclusion}
The aim of the project was to overcome the limitations of the conventional traffic systems where the time for which each lane is allowed to pass through was preset or fixed. The proposed system overcomes this problem by calculating/estimating this time in REAL-TIME depending upon the congestion in that corresponding lane thereby reducing the inflow of the traffic and ensuring proper flow of the vehicles in accordance with the dynamic traffic variations. This Project proves helpful in reducing the time wastage at the traffic signal whenever the traffic is less, since the legacy systems have preset timer values. Along with it, the overriding of the flow of the system by means of the IOT platform through cloud using android based user interface makes it easy for the traffic controller to control the signal from any place without the need for being present in person during emergency situations like – passing of ambulances, fire brigades, etc.

Finally, use of this system proves helpful as it is adaptive and flexible that auto adjusts depending upon the current traffic/congestion, thus overcoming many of the drawbacks of the current conventional traffic systems.

\pagebreak

\section{Future Works}
The proposed model is a 1st step towards making the controlling and handling of the traffic in large cities easier and adaptive. The system paves way for numerous possibilities that can be incorporated upon without changing much of the current prototype. Some of them are listed below,
\begin{itemize}
\item Current system is made to be constrained to a single traffic signal or circle. The system might be modified to handle multiple signals of the city through a single endpoint.
\item The linking of system to multiple signals together under a common control might demand the IOT system to be converted to wireless, since the wired connection and linking up all of them together through wired medium might be tedious.
\item Once the system is converted to support communication through wireless means, we can have a way to send the current traffic trends of all the different signals of a city to a common endpoint in cloud and then this data can be used to visualize the complete overview of the traffic in the city and can take necessary measures based on the inference arrived from the data visualization.
\item Also, with the dynamic estimation of green time, we can go on with other lane switching algorithm other than the current round robin method, if the traffic is very high such as – priority switching in which the system is switched to a lane with high congestion more frequently to allow more vehicles to pass through thereby ensuring smooth flow of congestion in all the lanes.
\end{itemize}

\pagebreak
