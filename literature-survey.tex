\section{Literature Survery}
A continuous increase in the number of vehicles being used by the masses, put forth an issue of increased traffic and congestion, which is to be addressed. Daily commute of people in and around the city, has been a problem due to inefficient management of the traffic in the city.

According to this article \cite{trafficcongestion}, it is estimated that, in our country, due to congestion, there is a annual loss of around Rs.60,000 crores which includes fuel wastage. Due to the traffic congestion, the average fuel mileage of a vehicle is only about 3.9kmpl and it also leads to wastage of time.

Here, Hussain T.M, et.all \cite{infrared}, describes an experimental infrared optical system that was designed to detect and monitor vehicular road traffic. The system used was developed and 1st tested in the laboratory using infrared laser sources and detectors in conjunction with computerized signal processing and correlation techniques. Preliminary road tests confirmed the system's capability to detect, monitor, and count the passage of vehicular road traffic.

In this paper \cite{multiloop}, S.S.M Ali presents an inductive loop vehicle detection system suitable for heterogeneous and less-lane disciplined traffic. A multiple loop system that is suitable for sensing vehicles in a heterogeneous and less-lane disciplined condition. Each loop has a unique resonance frequency, when a vehicle goes over a loop, the corresponding inductance and resonance frequency will change. This shift in frequency can be used to detect vehicles.

Xianbin Cao, et.all \cite{lowaltitude}, discusses about visual surveillance from low-altitude airborne platforms, that plays a key role in urban traffic surveillance. Moving vehicle detection and motion analysis are very important for such a system. This paper has two major contributions: First, to speed up feature extraction and to retain additional global features for higher classification accuracy. Second, to efficiently correlate vehicles across different frames for vehicle motion trajectories computation, thus achieving better performance with higher detection rate, lower false positive rate, and faster detection speed.

Based on the references from the above papers, we have come up with a more, simple model to control traffic and congestion effectively and co-ordinate emergency vehicles route as mentioned in the objective of the project.
